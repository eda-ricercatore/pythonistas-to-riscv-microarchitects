%	This LaTeX file is written by Zhiyang Ong as a template for creating presentation slides.

%	The MIT License (MIT)

%	Copyright (c) <2023> <Zhiyang Ong>

%	Permission is hereby granted, free of charge, to any person obtaining a copy of this software and associated documentation files (the "Software"), to deal in the Software without restriction, including without limitation the rights to use, copy, modify, merge, publish, distribute, sublicense, and/or sell copies of the Software, and to permit persons to whom the Software is furnished to do so, subject to the following conditions:

%	The above copyright notice and this permission notice shall be included in all copies or substantial portions of the Software.

%	THE SOFTWARE IS PROVIDED "AS IS", WITHOUT WARRANTY OF ANY KIND, EXPRESS OR IMPLIED, INCLUDING BUT NOT LIMITED TO THE WARRANTIES OF MERCHANTABILITY, FITNESS FOR A PARTICULAR PURPOSE AND NONINFRINGEMENT. IN NO EVENT SHALL THE AUTHORS OR COPYRIGHT HOLDERS BE LIABLE FOR ANY CLAIM, DAMAGES OR OTHER LIABILITY, WHETHER IN AN ACTION OF CONTRACT, TORT OR OTHERWISE, ARISING FROM, OUT OF OR IN CONNECTION WITH THE SOFTWARE OR THE USE OR OTHER DEALINGS IN THE SOFTWARE.

%	Email address: echo "cukj -wb- 23wU4X5M589 TROJANS cqkH wiuz2y 0f Mw Stanford" | awk '{ sub("23wU4X5M589","F.d_c_b. ") sub("Stanford","d0mA1n"); print $5, $2, $8; print " "; for (i=1; i<=1; i++) print "6\b"; print $9, $7, $6 }' | sed y/kqcbuHwM62z/gnotrzadqmC/ | tr 'q' ' ' | tr -d "\n" | tr -d 'ir' | tr y "\n"



%%%%%%%%%%%%%%%%%%%%%%%%%%%%%%%%%%%%%%%%%%%%%%
%	Preamble

%	Acknowledgement:
%		This is based on a template provided to me by Dott. Francesco Stefanni, from the University of Verona in January 2011.
%
%	Number the slides per section. This makes it easier to track the index of the slides (or number of slides) per section, as opposed to the cumulative number of slides. When I manually track the number of slides for a presentation, each time I refactor the set of slides, I would have to update the slide numbers. I want the computer to do this automatically. Hence, I shall not do this manually.



%	Use the Beamer package to create the presentation slides.
\documentclass[xcolor={usenames,dvipsnames},hyperref={hyperindex,bookmarks}]{beamer}
%	Background color: Set it to blue.
%\setbeamercolor{background canvas}{bg=blue}
%	\setbeamercolor{normal text}{bg=white,fg=yellow}
%\setbeamercolor{normal text}{fg=white}
%	\setbeamercolor{title}{fg=yellow,bg=white}
%\setbeamercolor{title}{fg=yellow}
%	\setbeamercolor{titlelike}{fg=yellow,bg=white}
%\setbeamercolor{block title alerted}{fg=white,bg=yellow}






%%%%%%%%%%%%%%%%%%%%%%%%%%%%%%%%%%%%%%%%%%%%%%
%	Import and Customize LaTeX packages.
\usepackage{beamerthemesplit}


%	Package for typesetting the following symbol: $\mathfrak{S}$
%\usepackage{amssymb}

%\mode<presentation>
%{ \usetheme{boxes} }

%	Select the presentation mode.
%	This is the line that requires the beamerthemeEsd.sty LaTeX style file.
\mode<presentation>{
	\usetheme[logos=true,pagenumbers=true,background=true]{Esd}
}
\setbeamercovered{transparent}
%\setbeamercovered{invisible}


%	Import package to facilitate typesetting of algorithms.
\usepackage{listings}

\lstset{
  language=Python,
  tabsize=4,
%  basicstyle=\ttfamily\color{black}\small,
  basicstyle=\ttfamily\color{black},
%  backgroundcolor=\color{lightgray},
%  backgroundcolor=\color{white},
  keywordstyle=\color{Purple}\bfseries,
  identifierstyle=\color{OliveGreen},
  commentstyle=\color{Gray}\itshape,
  stringstyle=\color{CarnationPink},
  showstringspaces=false,
  showtabs=false,
  showspaces=false
}


\definecolor{lightgray}{gray}{0.95}
\font\emailtt=cmtt9

%	Set up configuration for hyperlinks.
%\usepackage[pdftex]{hyperref}	-- Option clash
\hypersetup{
    pdftitle={{\it Python}istas become {RISC-V} microarchitects},     % title
    pdfauthor={Zhiyang Ong},                 % author
    pdfsubject={{RISC-V} Karachi Meetup 2023}, % subject of the document
    pdfcreator={Creator},                           % creator of the document
    pdfproducer={dvipdft},                          % producer of the document
% Modified by Zhiyang Ong on Feb 7, 2011 to improve the way hyperlinks are colored in these presentation slides
	pdfkeywords={LaTeX, graphics, color},
%    pdfkeywords={C, C++, programming style},        % list of keywords
%
%    bookmarks=true,         % show bookmarks bar?
    unicode=false,          % non-Latin characters in Acrobats bookmarks
    pdftoolbar=true,        % show Acrobats toolbar?
    pdfmenubar=true,        % show Acrobats menu?
    pdffitwindow=false,     % window fit to page when opened
% Modified by Zhiyang Ong on Feb 7, 2011 to improve the way hyperlinks are colored in these presentation slides
	pdfpagemode=UseOutlines,bookmarks, bookmarksopen,
	pdfstartview=FitH, colorlinks, linkcolor=blue, citecolor=blue, urlcolor=red,
%    pdfstartview={Fit},    % fits the width of the page to the window
    pdfnewwindow=true,      % links in new window
% Modified by Zhiyang Ong on Feb 7, 2011 to improve the way hyperlinks are colored in these presentation slides
	colorlinks=red,        % false: boxed links; true: colored links
	linkcolor=red,          % color of internal links
%    colorlinks=false,        % false: boxed links; true: colored links
%    linkcolor=red,          % color of internal links
    citecolor=green,        % color of links to bibliography
    filecolor=magenta,      % color of file links
    urlcolor=red,           % color of external links
    pdfpagemode=FullScreen
    %
    %pdfpagelabels=false
}







%%%%%%%%%%%%%%%%%%%%%%%%%%%%%%%%%%%%%%%%%%%%%%
%%%%%%%%%%%%%%%%%%%%%%%%%%%%%%%%%%%%%%%%%%%%%%
%%%%%%%%%%%%%%%%%%%%%%%%%%%%%%%%%%%%%%%%%%%%%%
%%%%%%%%%%%%%%%%%%%%%%%%%%%%%%%%%%%%%%%%%%%%%%
%%%%%%%%%%%%%%%%%%%%%%%%%%%%%%%%%%%%%%%%%%%%%%
%%%%%%%%%%%%%%%%%%%%%%%%%%%%%%%%%%%%%%%%%%%%%%
%%%%%%%%%%%%%%%%%%%%%%%%%%%%%%%%%%%%%%%%%%%%%%


%	Quantum Model Checking Is Not Evil: It Is Mandatory For Quantum Robots


%	First slide of the presentation
\title[{RISC-V} Karachi Meetup 2023]
%	RISC-V Karachi Meetup 2023
%	Venues: 
%	Host: Mr. Zeeshan Rafique
%	RISC-V Foundation: RISC-V APAC
%	from September 29, 2023 @ 23:30 p.m. CDT till September 30, 2023 @ 07:00 a.m.
%	Micro Electronics Research Lab, Faculty of Engineering and Technology, Usman Institute of Technology
%	Led by Prof. Ali Ahmed
{\huge 
Helping {\it Python}istas Become Microarchitects}
\subtitle{Using Jupyter Notebooks and CIRCT/MLIR/LLVM}
\author{Zhiyang Ong}
\institute{
	Department of Electrical and Computer Engineering \\
	College of Engineering,\\
	Texas A\&M University \\
	College Station, TX
}
\date{\today}	% (optional)
\subject{Subject Title}







%%%%%%%%%%%%%%%%%%%%%%%%%%%%%%%%%%%%%%%%%%%%%%
%	Do nothing in this section of the LaTeX document

\begin{document}

\begin{frame}
\titlepage
\end{frame}



%	Table of Contents
\AtBeginSection[]		% Do nothing for \subsection*
{
	\begin{frame}
%		\frametitle{\textcolor{yellow}{Table of Contents}}
		\frametitle{Table of Contents}
%		\textcolor{yellow}{\tableofcontents[currentsection]}
		\tableofcontents[currentsection,currentsubsection]
	\end{frame}
}

\AtBeginSubsection[]		% Do nothing for \subsection*
{
\begin{frame}
\tableofcontents[currentsection,currentsubsection]
\end{frame}
}

\section*{Outline}
\begin{frame}
\tableofcontents
\end{frame}

%	IMPORTANT: Note that entries for the Table of Contents are indicated by sections and subsections.





%%%%%%%%%%%%%%%%%%%%%%%%%%%%%%%%%%%%%%%%%%%%%%
%	Section One
\section{Problems in Computer Architecture}

%%%%%%%%%%%%%%%%%%%%%%%%%%%%%%%%%%%%%%%%%%%%%%
%	Problems in Computer Architecture
%	Slide 1

\begin{frame}
	\frametitle{Problems in Computer Architecture}
	\framesubtitle{Specifically with General-Purpose Processor Architectures}
	
	\begin{itemize} %\itemsep -4pt
	\item {\bf Memory Wall} [Wulf1995] [Hennessy1990] [Horowitz2023] [Solihin2002]
	\item {\bf End of Dennard's scaling} [Dennard1974] [Haensch2006] [Chen2006] [Dennard2007] [Calhoun2008] [Iwai2009] and {\bf Power Wall} [Keshavarzi2007]
	\item {\bf Dark Silicon} [Esmaeilzadeh2011] [Esmaeilzadeh2012] [Rahmani2017] [Hurson2018]
	
	\end{itemize}

\end{frame}




%%%%%%%%%%%%%%%%%%%%%%%%%%%%%%%%%%%%%%%%%%%%%%
%	Problems in Computer Architecture
%	Slide 2

\begin{frame}
	\frametitle{Problems in Computer Architecture}
	\framesubtitle{Specifically with General-Purpose Processor Architectures}
	
	\begin{figure}
		\centering
		\includegraphics[height=2.1in]{./pics/ProblemsInComputerArchitecture}
		\caption{Plot of the performance of general-purpose processors over time, from 1980 till the late 2010s [Hennessy2018]}
	\end{figure}
\end{frame}







%%%%%%%%%%%%%%%%%%%%%%%%%%%%%%%%%%%%%%%%%%%%%%
%	Section Two
\section{Section Two}


%%%%%%%%%%%%%%%%%%%%%%%%%%%%%%%%%%%%%%%%%%%%%%
%	Subsection 2.1
\subsection{Subsection 2.1}


%%%%%%%%%%%%%%%%%%%%%%%%%%%%%%%%%%%%%%%%%%%%%%
%	Slide 3
\begin{frame}
	\frametitle{Slide Title 3}
	\framesubtitle{Slide Subtitle 3}
	
	\begin{columns}[t]			% Contents are top vertically aligned
		\begin{column}[T]{5cm}	% Each column can also be its own environment
		
		Statement 1. \\
		\ \\
		Statement 2. \\
		\ \\
		Statement 3.
		\end{column}
		
		\begin{column}[T]{5cm}	% Alternative top-align that's better for graphics
			\begin{figure}
			\centering
			\includegraphics[height=2in]{./pics/my_figure}
			\caption{My caption \cite{Petroski1992,Kopka2004}}
			\end{figure}
		\end{column}
	\end{columns}
\end{frame}

%%%%%%%%%%%%%%%%%%%%%%%%%%%%%%%%%%%%%%%%%%%%%%
%	Subsection 2.2
\subsection{Subsection 2.2}

%%%%%%%%%%%%%%%%%%%%%%%%%%%%%%%%%%%%%%%%%%%%%%
%	Slide 4
\begin{frame}
	\frametitle{Slide Title 4}
	\framesubtitle{Slide Subtitle 4.}
	
	Statement 1 \cite{vanDongen2012}. \\
	\ \\
	Statement 2. \\
	\ \\
	Statement 3.
\end{frame}

%%%%%%%%%%%%%%%%%%%%%%%%%%%%%%%%%%%%%%%%%%%%%%
%	Section 3
\section{Section 3}

%%%%%%%%%%%%%%%%%%%%%%%%%%%%%%%%%%%%%%%%%%%%%%
%	Slide 5
\begin{frame}
	\frametitle{Slide Title 5}
	\framesubtitle{Slide Subtitle 5.}
	
	\begin{figure}
		\centering
		\includegraphics[height=2.6in]{./pics/my_figure}
		\caption{Caption \#2}
	\end{figure}
\end{frame}




%%%%%%%%%%%%%%%%%%%%%%%%%%%%%%%%%%%%%%%%%%%%%
%	Slide 6
%	References
{\linespread{1}
\begin{frame}
	\frametitle{References}
	\bibliographystyle{plain}
	\bibliography{references/references}
\end{frame}
}
%%%%%%%%%%%%%%%%%%%%%%%%%%%%%%%%%%%%%%%%%%%%%
\end{document}